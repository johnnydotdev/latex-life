\documentclass[letterpaper, 10pt]{article}
\begin{document}

\title{CS 374 Fall 2015 well\\
Homework 0\\
Problem 1
}

\author{Johnny Chang \\ jychang3}
\maketitle

\newpage
    1. Prove that for any positive integer $n$ and any set $X \subseteq \{1, 2, ..., 2n\}$ such that $|X| = n + 1$, there exist two distinct elements $a, b$ in $X$ such that $a$ is a multiple of $b$.\\

    We know from the original problem statement that any integer can be written as the product of an odd number and a power of 2.\\

    For the set $X$, let us create at most $n$ bins that correspond to the odd numbers in $X$. There are $n$ bins at most because there are $2n$ elements in the set $X$ and odd numbers are half of them. $2n/2 = n$. \\

    Each member of $X$ can be represented as the product of an odd number and a power of 2. This means each member of $X$ can be assigned to a bin based on the product of the \emph{largest odd number corresponding to the bins} that fits into that member and a power of 2. The odd number will be its bin.\\

    Note that selecting $n+1$ members from $n$ bins means that we will have to draw from one bin more than one time, by the Pigeonhole Principle. Drawing from a bin more than once means that we will choose a multiple of a number already chosen from that bin.\\

    Thus, there exist two distinct elements $a, b$ in $X$ such that $a$ is a multiple of $b$, for any positive integer $n$ and any set $X \subseteq \{1, 2, ..., 2n\}$ such that $|X| = n + 1$.\\

    Citations:
    \begin{itemize}
        \item Nitesh Nath
        \item Alek Festekjian
        \item Margaret Fleck's book
    \end{itemize}

\end{document}
